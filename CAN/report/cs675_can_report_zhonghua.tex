\documentclass[11pt, oneside]{article}   	% use "amsart" instead of "article" for AMSLaTeX format
\usepackage{geometry}                		% See geometry.pdf to learn the layout options. There are lots.
\geometry{letterpaper}                   		% ... or a4paper or a5paper or ... 
%\geometry{landscape}                		% Activate for for rotated page geometry
%\usepackage[parfill]{parskip}    		% Activate to begin paragraphs with an empty line rather than an indent
\usepackage{graphicx}				% Use pdf, png, jpg, or eps§ with pdflatex; use eps in DVI mode
								% TeX will automatically convert eps --> pdf in pdflatex		
\usepackage{amssymb}
\usepackage{amsmath}


\title{CS675 Project 1: P2P Architecture}
\author{Zhonghua Xi}
%\date{}							% Activate to display a given date or no date

\begin{document}
\maketitle

\section{Design}
\subsection{Peer}
A peer node is both a client and server which provides both CAN node service and bootstrap service.

\subsection{Routing}
When the coordinate of a target point $p$ is given, the CAN system routes the message in the following manner.
If current peer $P$'s zones contain $p$, $P$ processes the request. Otherwise $P$ forwards the request to its neighbor who is closest to $p$. The distance $D$ between a node $N$ and a point $p$ is defined as: 
\begin{equation}
D(N,p) = 
\begin{cases}
    0, 										& \text{if } N \text{'s zones contain } p \\
 	\min\limits_{i}(\lVert N.zone_i.center - p \rVert),      & \text{otherwise}
\end{cases}
\end{equation}
The above distance metic guarantees the message will be routed to the node whose zones contain $p$.


\section{Implementation}

\subsection{Language}
Java is chosen to implement this project since Java provides a lot of tools to deal with network communications.

\subsection{Networking Technique}
Java RMI is chosen as the networking technique. RMI can be regarded as an object-oriented version RPC which provides high level abstraction and low level of complexity.
Request and response become function parameters and returned value. 
The client will be blocked (by default) until receives the response. 
Compare to a socket based implementation, in which request and response are not deeply-related, the client code in RMI is more semantically meaningful and more elegant.

\subsection{Peer}
When a peer is about to start, a bootstrap server (ip/hostname and name) must be provided. Otherwise, that peer will think it is the only node in CAN and will occupy the entire zone.
The peer will provide both Node service (join, insert, search) and Bootstrap (get active node list). These two services are bind to different names.

\subsubsection{Bootstrap Service}
For a bootstrap server, it has only one remote method:

\subsubsection{Node Service}




\end{document}  